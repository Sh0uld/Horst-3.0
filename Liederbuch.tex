\documentclass[a5paper,
              %paper=landscape,
              12pt,
              pagesize,
              typearea,
              titlepage,
              DIV=10,
              twoside
              ]{scrbook}


\usepackage[utf8]{inputenc}
\usepackage[T1]{fontenc}
\usepackage[ngerman]{babel}
\usepackage{gensymb}

%Für Akkord-Grafiken
\usepackage{subfig}
\usepackage{tikz}        
              
%Zusätzliche Beschnittzugabe zum DinA6 Format dazu
%\usepackage[cam, noinfo, width=15.4truecm,height=11.1truecm, center]{crop}

%Überschriftenabstände
\usepackage{titlesec}

%Kopf- und Fußzeilen
\usepackage{fancyhdr}

%Schrift(en)
\usepackage[sfdefault]{AlegreyaSans}

%Gitarrenakkorde
\usepackage{guitar}

%Index
%\usepackage{makeidx}
\usepackage{imakeidx}
\makeindex[columns=2, title=Inhaltsverzeichnis, options= -s index-style.ist]

%PDF-Dateien einbinden 
\usepackage{pdfpages}

%Layoutgeraffel
%may the latex masters be gentle and hope that they won't take this clusterf**k personal - but it works
%setting up the layout spacing 
\areaset[10mm]{130 mm}{179 mm}
\topmargin=-21.4mm 
\headheight=5.5mm
\headsep=2.5mm
\footskip=6mm
\setlength{\parindent}{0em}
\setcounter{secnumdepth}{0}

\setlength\parskip{1\baselineskip plus 1\baselineskip minus 0.15\baselineskip}

\titleformat{\chapter}[display]  
{\normalfont\huge\bfseries}{\chaptertitlename\ \thechapter}{15pt}{\Huge}   

\titleformat{\section}[display]
{\normalfont\normalfont\bfseries}{\chaptertitlename\ \thechapter}{15pt}{\Large}

\titlespacing*{\chapter}{0pt}{-15mm}{5mm}
\titlespacing*{\section}{0mm}{0mm}{-5mm}
\fancypagestyle{plain}{\fancyhf{}%
  \renewcommand{\headrulewidth}{0pt}
}

%Separiert Kopfzeile und Fußzeile einer Seite mit einem horizontalen Strich
\renewcommand{\headrulewidth}{0.4pt}% default is 0pt
%\renewcommand{\footrulewidth}{0.4pt}% default is 0pt

%Akkord-Schrift (bzw. Änderungen zum Normalfont)
\usepackage{courierten}
\renewcommand{\guitarPreAccord}{\fontencoding{T1}\fontfamily{\ttdefault}\selectfont\bfseries}

%%
%%Befehlesetup
%%
\newcommand{\whzl}{\raisebox{-1mm}{\hspace{0.3mm}\rule{0.3mm}{4mm}\hspace{0.3mm}\rule{0.1mm}{4mm}\hspace{0.2mm}}:}
\newcommand{\whzr}{:\raisebox{-1mm}{\hspace{0.2mm}\rule{0.1mm}{4mm}\hspace{0.3mm}\rule{0.3mm}{4mm}}}
\newcommand{\liedweiter}{ \AddToShipoutPictureFG*{%
                            \AtTextLowerLeft{%
                              \makebox[\textwidth][r]{
                                  \includegraphics[scale=0.025]{icons/wegzeichen_naechste.pdf}
                              }%
                            }%
                          }
                          \newpage 
}
\newcommand{\WorteUndWeiseSetup}{\small \textit{\WorteUndWeise}}
\newcommand{\ErlklaerenderTextSetup}[1]{\small \textit{#1}}

%%
%% Seiten-Setups
%%
\newcommand{\LiedSetup}{ 
                        \noindent
                        \section{\Titel}
                        \fancyhead[RE,LO]{\nouppercase{\Titel}}
%                        \fancyhead[RO,LE]{\small{\nouppercase{\Interpret}}} 
                        \fancyfoot[LE,RO]{\small{\nouppercase{\Referenz}}} 
                        \fancyfoot[LO,RE]{\thepage}
}\newcommand{\AkkordeSetup}{ 
%                        \noindent
%                        \section{\Titel}
                        \fancyhead[RE,LO]{\nouppercase{\Titel}}
%                        \fancyhead[RO,LE]{\small{\nouppercase{\Interpret}}} 
%                        \fancyfoot[LE,RO]{\small{\nouppercase{\Referenz}}} 
                        \fancyfoot[LO,RE]{\thepage}
}
\newcommand{\EinSchnellesLiedSetup}{
                        \noindent
                        \section{\Titel}
                        \fancyhead[RE,LO]{\nouppercase{\Titel}}
%                        \fancyhead[RO,LE]{\small{\nouppercase{\Interpret}}} 
                        \fancyfoot[LE,RO]{\small{\nouppercase{\Referenz}}} 
                        \fancyfoot[LO,RE]{\thepage}
                        \addcontentsline{lof}{section}{\Titel}
}   

%%
%% Akkorde
%% Code from chords.sty
%% Modified and extended by Paddy
%%

% Counters
\newcounter{chords-string}
\newcounter{chords-fret}

% Variables
\newcommand{\chordreset}{
  \def\chordtuning{E,A,D,G,B,E}
  \def\chordfretstart{1}
  \def\chordfretend{4}
}
\chordreset

% The chord environmant
% \begin{chord}
% \begin{chord}[D,G,D,G,H,E]
\newenvironment{chord}
{
  % Chord name
  % \chordname <chordname>
  \newcommand{\chordname}[1]{
    \draw node[title] at (0,0) {##1};
  }

  % A single note
  % \single <string> <fret> <finger>
  \newcommand{\single}[3]{
    \draw node[single] at (##1,##2) {##3};
  }
  % A bar
  % \bar <startstring> <fret> <finger>
  \renewcommand{\bar}[3]{
    \draw[bar] (##1,##2) -- node[midway] {##3} (1,##2);
  }
  % No strike
  % \nostrike <string>
  \newcommand{\nostrike}[1]{
    \draw[nostrike] (##1,\chordfretstart-.5) +(-135:.2cm) -- +(45:.2cm);
    \draw[nostrike] (##1,\chordfretstart-.5) +(135:.2cm) -- +(-45:.2cm);
  }
  \begin{tikzpicture}[
    scale=0.85,
    title/.style={draw,font=\bfseries},
    single/.style={draw,circle,fill=white},
    bar/.style={cap=round,double,double distance=18pt},
    nostrike/.style={line width=.8mm},
    cm={0,-0.8,1,0,(0,0)}
  ]
  \setcounter{chords-string}{6}
  \foreach \tuning in \chordtuning
  {
    \node at (\value{chords-string},\chordfretstart-1) {\tuning};
    \addtocounter{chords-string}{-1}
  }
  \draw[yshift=-0.5cm] (1,\chordfretstart) grid (6,\chordfretend+1);

  \foreach \fret in {\chordfretstart,...,\chordfretend}
  {
    \setcounter{chords-fret}\fret
    \draw node at (0,\fret) {\Roman{chords-fret}};
  }
}
{
  \end{tikzpicture}
}


%%
%% Liederbuch Start
%%                    

%\makeindex

\begin{document}
  \pagestyle{empty}
  

    %Cover 
    \includepdf[scale=1.1]{bilder/klappentext_modern.pdf}
    %leerer Klappentext
    \cleardoublepage
    
    %Hier ein Vorwort, welches nicht im Inhaltsverzeichnis auftaucht
    \chapter*{Vorwort}
      Aus der tiefen Überzeugung heraus,\\
      dass jeder Mensch\\
      einen Horst braucht.\\
      
      \par\noindent\rule{\textwidth}{0.4pt}
      
      Nur für den internen Gebrauch im\\
      Bund der Pfadfinderinnen und Pfadfinder e.V.
      
      \par\noindent\rule{\textwidth}{0.4pt}
      
      Herausgeber: BdP Stamm Greutungen e.V., Wiesbaden
      
      \newpage

    %Seitenrahmen einrichtung
    \pagenumbering{arabic}
    \fancyhf{}
    \pagestyle{fancy}
    
    %Das Ein schnelles Lied! fasst alle entsprechend markierten Lieder in einem weiteren Inhaltsverzeichnis zusammen
    \renewcommand\listfigurename{Ein schnelles Lied!}
    %\addcontentsline{toc}{chapter}{\listfigurename{}}
    \listoffigures
    
    \newcommand{\Lied}[1]{
        \include{Horst-3.0-Liedtexte/#1}
    }
    
    %%
    %% Hier Lieder einfügen
    %% \Lied{<Dateiname ohne .tex>}
    %%
    
    \Lied{Abends}
    \Lied{Abends treten Elche}
    \Lied{Abschied im Herbstwind}
    \Lied{Abschied von der Meute}
    \Lied{Ach, Glöckelein}
    
    \Lied{Maienbaum}
    
    %%
    %% Ende der Lieder
    %%
    
    % Akkord-Seiten einfügen
    %%
%% Akkorde
%% Code from chords.sty
%% Modified and extended by Paddy
%%

% Counters
\newcounter{chords-string}
\newcounter{chords-fret}

% Variables
\newcommand{\chordreset}{
  \def\chordtuning{E,A,D,G,B,E}
  \def\chordfretstart{1}
  \def\chordfretend{4}
}
\chordreset

% The chord environmant
% \begin{chord}
% \begin{chord}[D,G,D,G,H,E]
\newenvironment{chord}
{
  % Chord name
  % \chordname <chordname>
  \newcommand{\chordname}[1]{
    \draw node[title] at (0,0) {##1};
  }

  % A single note
  % \single <string> <fret> <finger>
  \newcommand{\single}[3]{
    \draw node[single] at (##1,##2) {##3};
  }
  % A bar
  % \bar <startstring> <fret> <finger>
  \renewcommand{\bar}[3]{
    \draw[bar] (##1,##2) -- node[midway] {##3} (1,##2);
  }
  % No strike
  % \nostrike <string>
  \newcommand{\nostrike}[1]{
    \draw[nostrike] (##1,\chordfretstart-.5) +(-135:.2cm) -- +(45:.2cm);
    \draw[nostrike] (##1,\chordfretstart-.5) +(135:.2cm) -- +(-45:.2cm);
  }
  \begin{tikzpicture}[
    scale=0.85,
    title/.style={draw,font=\bfseries},
    single/.style={draw,circle,fill=white},
    bar/.style={cap=round,double,double distance=18pt},
    nostrike/.style={line width=.8mm},
    cm={0,-0.8,1,0,(0,0)}
  ]
  \setcounter{chords-string}{6}
  \foreach \tuning in \chordtuning
  {
    \node at (\value{chords-string},\chordfretstart-1) {\tuning};
    \addtocounter{chords-string}{-1}
  }
  \draw[yshift=-0.5cm] (1,\chordfretstart) grid (6,\chordfretend+1);

  \foreach \fret in {\chordfretstart,...,\chordfretend}
  {
    \setcounter{chords-fret}\fret
    \draw node at (0,\fret) {\Roman{chords-fret}};
  }
}
{
  \end{tikzpicture}
}

    
    \pagestyle{empty}
    
    %Inhaltsverzeichnis
    \printindex
    
\end{document}