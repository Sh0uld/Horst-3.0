\documentclass[a5paper,
              %paper=landscape,
              12pt,
              pagesize,
              typearea,
              titlepage,
              DIV=10,
              twoside
              ]{scrbook}


\usepackage[utf8]{inputenc}
\usepackage[T1]{fontenc}
\usepackage[ngerman]{babel}

%Für Akkord-Grafiken
\usepackage{subfig}
\usepackage{tikz}
              
              
%Zusätzliche Beschnittzugabe zum DinA6 Format dazu
%\usepackage[cam, noinfo, width=15.4truecm,height=11.1truecm, center]{crop}

%Überschriftenabstände
\usepackage{titlesec}

%Kopf- und Fußzeilen
\usepackage{fancyhdr}

%Schrift(en)
\usepackage[sfdefault]{AlegreyaSans}

%Gitarrenakkorde
\usepackage{guitar}

%Index
\usepackage{makeidx}

%Strophennummerierung
%\usepackage{enumitem}

%PDF-Dateien einbinden 
\usepackage{pdfpages}

%Layoutgeraffel
%may the latex masters be gentle and hope that they won't take this clusterf**k personal - but it works
%setting up the layout spacing 
\areaset[10mm]{130 mm}{179 mm}
\topmargin=-21.4mm 
\headheight=5.5mm
\headsep=2.5mm
\footskip=6mm
\setlength{\parindent}{0em}
%\setlist[enumerate]{topsep=0pt, itemsep=5pt, partopsep=-10pt, parsep=5pt, wide, labelwidth=!, labelindent=0pt, label=\bfseries{\arabic*.}}
\setcounter{secnumdepth}{0}

\setlength\parskip{1\baselineskip plus 1\baselineskip minus 0.15\baselineskip}

\titleformat{\chapter}[display]  
{\normalfont\huge\bfseries}{\chaptertitlename\ \thechapter}{15pt}{\Huge}   

\titleformat{\section}[display]
{\normalfont\normalfont\bfseries}{\chaptertitlename\ \thechapter}{15pt}{\Large}

\titlespacing*{\chapter}{0pt}{-15mm}{5mm}
\titlespacing*{\section}{0mm}{0mm}{-5mm}
\fancypagestyle{plain}{\fancyhf{}%
  \renewcommand{\headrulewidth}{0pt}
}

%Separiert Kopfzeile und Fußzeile einer Seite mit einem horizontalen Strich
\renewcommand{\headrulewidth}{0.4pt}% default is 0pt
%\renewcommand{\footrulewidth}{0.4pt}% default is 0pt

%Befehlesetup
\newcommand{\whzl}{\raisebox{-1mm}{\hspace{0.3mm}\rule{0.3mm}{4mm}\hspace{0.3mm}\rule{0.1mm}{4mm}\hspace{0.2mm}}:}
\newcommand{\whzr}{:\raisebox{-1mm}{\hspace{0.2mm}\rule{0.1mm}{4mm}\hspace{0.3mm}\rule{0.3mm}{4mm}}}
\newcommand{\liedweiter}{ \AddToShipoutPictureFG*{%
                            \AtTextLowerLeft{%
                              \makebox[\textwidth][r]{
                                  \includegraphics[scale=0.025]{icons/wegzeichen_naechste.pdf}
                              }%
                            }%
                          }
                          \newpage 
}

%%
%% Seiten-Setups
%%
\newcommand{\LiedSetup}{ 
                        \noindent
                        \section{\Titel}
                        \fancyhead[RE,LO]{\nouppercase{\Titel}}
%                        \fancyhead[RO,LE]{\small{\nouppercase{\Interpret}}} 
%                        \fancyfoot[LE,RO]{\small{\nouppercase{\Referenz}}} 
                        \fancyfoot[LO,RE]{\thepage}
}\newcommand{\AkkordeSetup}{ 
%                        \noindent
%                        \section{\Titel}
                        \fancyhead[RE,LO]{\nouppercase{\Titel}}
%                        \fancyhead[RO,LE]{\small{\nouppercase{\Interpret}}} 
%                        \fancyfoot[LE,RO]{\small{\nouppercase{\Referenz}}} 
                        \fancyfoot[LO,RE]{\thepage}
}
\newcommand{\EinSchnellesLiedSetup}{
                        \noindent
                        \section{\Titel}
                        \fancyhead[RE,LO]{\nouppercase{\Titel}}
%                        \fancyhead[RO,LE]{\small{\nouppercase{\Interpret}}} 
%                        \fancyfoot[LE,RO]{\small{\nouppercase{\Referenz}}} 
                        \fancyfoot[LO,RE]{\thepage}
                        \addcontentsline{lof}{section}{\Titel}
}   

%%
%% Akkorde
%% Code from chords.sty
%% Modified and extended by Paddy
%%

% Counters
\newcounter{chords-string}
\newcounter{chords-fret}

% Variables
\newcommand{\chordreset}{
  \def\chordtuning{E,A,D,G,B,E}
  \def\chordfretstart{1}
  \def\chordfretend{4}
}
\chordreset

% The chord environmant
% \begin{chord}
% \begin{chord}[D,G,D,G,H,E]
\newenvironment{chord}
{
  % Chord name
  % \chordname <chordname>
  \newcommand{\chordname}[1]{
    \draw node[title] at (0,0) {##1};
  }

  % A single note
  % \single <string> <fret> <finger>
  \newcommand{\single}[3]{
    \draw node[single] at (##1,##2) {##3};
  }
  % A bar
  % \bar <startstring> <fret> <finger>
  \renewcommand{\bar}[3]{
    \draw[bar] (##1,##2) -- node[midway] {##3} (1,##2);
  }
  % No strike
  % \nostrike <string>
  \newcommand{\nostrike}[1]{
    \draw[nostrike] (##1,\chordfretstart-.5) +(-135:.2cm) -- +(45:.2cm);
    \draw[nostrike] (##1,\chordfretstart-.5) +(135:.2cm) -- +(-45:.2cm);
  }
  \begin{tikzpicture}[
    scale=0.85,
    title/.style={draw,font=\bfseries},
    single/.style={draw,circle,fill=white},
    bar/.style={cap=round,double,double distance=18pt},
    nostrike/.style={line width=.8mm},
    cm={0,-0.8,1,0,(0,0)}
  ]
  \setcounter{chords-string}{6}
  \foreach \tuning in \chordtuning
  {
    \node at (\value{chords-string},\chordfretstart-1) {\tuning};
    \addtocounter{chords-string}{-1}
  }
  \draw[yshift=-0.5cm] (1,\chordfretstart) grid (6,\chordfretend+1);

  \foreach \fret in {\chordfretstart,...,\chordfretend}
  {
    \setcounter{chords-fret}\fret
    \draw node at (0,\fret) {\Roman{chords-fret}};
  }
}
{
  \end{tikzpicture}
}

%%
%% Liederbuch Start
%%                    

\makeindex

\begin{document}
  \pagestyle{empty}
  

    %Cover 
    \includepdf[scale=1.1]{bilder/klappentext_modern.pdf}
    %leerer Klappentext
    \cleardoublepage
    
    %Hier ein Vorwort, welches nicht im Inhaltsverzeichnis auftaucht
    \chapter*{Vorwort}
      Hi,

      ich bin das Vorwort von dem Liederbuchtemplate das du gerade geöffnet hast.
      Hier könntest du dich allen zukünftigen Lesern deines Liederbuches vorstellen und mitteilen was für großartige Abende sie damit verbringen können.
      
      Gezeichnet,

      dein Liederbuchtemplate

      \newpage

    %Seitenrahmen einrichtung
    \pagenumbering{arabic}
    \fancyhf{}
    \pagestyle{fancy}
    
    %Das Ein schnelles Lied! fasst alle entsprechend markierten Lieder in einem weiteren Inhaltsverzeichnis zusammen
    \renewcommand\listfigurename{Ein schnelles Lied!}
    %\addcontentsline{toc}{chapter}{\listfigurename{}}
    \listoffigures
    \printindex
    
    % Dieses Template enhält nur ein einziges Lied. Wegen Liedrechten und solchen Dingen
    \def\Titel{Die Gedanken sind frei}
\def\Interpret{Deutsches Volkslied}
\def\Referenz{Möglicher Querverweis auf ein gebräuchliches Liederbuch deiner Wahl}

%Nur eine Variante auswählen! Die andere mit % auskommentieren oder löschen.
%\LiedSetup{}               %Nur Inhaltsverzeichnis
\EinSchnellesLiedSetup{}   %Inhaltsverzeichnis + Ein schnelles Lied!

\begin{guitarMagic}

    Die Ge[A]danken sind frei, wer [E7]kann sie er[A]raten? \\
    Sie fliehen vorbei wie [E7]nächtliche [A]Schatten.\\
    Kein [E]Mensch kann sie [A]wissen, kein [E]Jäger er[A]schießen
    mit [D]Pulver und [A]Blei.\\
    Die Ge[E]danken sind [A]frei!

    Ich denke, was ich will und was mich beglücket,\\
    doch alles in der Still’
    und wie es sich schicket.\\
    Mein Wunsch und Begehren
    kann niemand verwehren,
    es bleibet dabei:\\
    Die Gedanken sind frei!

    Und sperrt man mich ein
    im finsteren Kerker,\\
    das alles sind rein
    vergebliche Werke,\\
    denn meine Gedanken
    zerreißen die Schranken
    und Mauern entzwei.\\
    Die Gedanken sind frei!

    Drum will ich auf immer
    den Sorgen entsagen\\
    und will mich auch nimmer
    mit Grillen mehr plagen.\\
    Man kann ja im Herzen
    stets lachen und scherzen
    und denken dabei:\\
    Die Gedanken sind frei!

    Ich liebe den Wein,
    mein Mädchen vor allen,\\
    sie tut mir allein
    am besten gefallen.

    Ich bin nicht alleine
    bei meinem Glas Weine,
    mein Mädchen dabei.\\
    Die Gedanken sind frei!
    
\end{guitarMagic}

    
    % Akkord-Seiten einfügen
    %%
%% Akkorde
%% Code from chords.sty
%% Modified and extended by Paddy
%%

% Counters
\newcounter{chords-string}
\newcounter{chords-fret}

% Variables
\newcommand{\chordreset}{
  \def\chordtuning{E,A,D,G,B,E}
  \def\chordfretstart{1}
  \def\chordfretend{4}
}
\chordreset

% The chord environmant
% \begin{chord}
% \begin{chord}[D,G,D,G,H,E]
\newenvironment{chord}
{
  % Chord name
  % \chordname <chordname>
  \newcommand{\chordname}[1]{
    \draw node[title] at (0,0) {##1};
  }

  % A single note
  % \single <string> <fret> <finger>
  \newcommand{\single}[3]{
    \draw node[single] at (##1,##2) {##3};
  }
  % A bar
  % \bar <startstring> <fret> <finger>
  \renewcommand{\bar}[3]{
    \draw[bar] (##1,##2) -- node[midway] {##3} (1,##2);
  }
  % No strike
  % \nostrike <string>
  \newcommand{\nostrike}[1]{
    \draw[nostrike] (##1,\chordfretstart-.5) +(-135:.2cm) -- +(45:.2cm);
    \draw[nostrike] (##1,\chordfretstart-.5) +(135:.2cm) -- +(-45:.2cm);
  }
  \begin{tikzpicture}[
    scale=0.85,
    title/.style={draw,font=\bfseries},
    single/.style={draw,circle,fill=white},
    bar/.style={cap=round,double,double distance=18pt},
    nostrike/.style={line width=.8mm},
    cm={0,-0.8,1,0,(0,0)}
  ]
  \setcounter{chords-string}{6}
  \foreach \tuning in \chordtuning
  {
    \node at (\value{chords-string},\chordfretstart-1) {\tuning};
    \addtocounter{chords-string}{-1}
  }
  \draw[yshift=-0.5cm] (1,\chordfretstart) grid (6,\chordfretend+1);

  \foreach \fret in {\chordfretstart,...,\chordfretend}
  {
    \setcounter{chords-fret}\fret
    \draw node at (0,\fret) {\Roman{chords-fret}};
  }
}
{
  \end{tikzpicture}
}

    

    \pagestyle{empty}


    %Das Inhaltsverzeichnis  
    \tableofcontents
    \newpage


    %Leerer hinterer Klappentext
    \cleardoublepage
    %Letzte Seite
    \includepdf[scale=1.1]{bilder/klappentext_modern.pdf}
    
\end{document}

