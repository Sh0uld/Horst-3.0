\def\Titel{Die Gedanken sind frei}
\def\Interpret{Deutsches Volkslied}
\def\Referenz{Möglicher Querverweis auf ein gebräuchliches Liederbuch deiner Wahl}

%Nur eine Variante auswählen! Die andere mit % auskommentieren oder löschen.
%\LiedSetup{}               %Nur Inhaltsverzeichnis
\EinSchnellesLiedSetup{}   %Inhaltsverzeichnis + Ein schnelles Lied!

\begin{guitarMagic}

    Die Ge[A]danken sind frei, wer [E7]kann sie er[A]raten? \\
    Sie fliehen vorbei wie [E7]nächtliche [A]Schatten.\\
    Kein [E]Mensch kann sie [A]wissen, kein [E]Jäger er[A]schießen
    mit [D]Pulver und [A]Blei.\\
    Die Ge[E]danken sind [A]frei!

    Ich denke, was ich will und was mich beglücket,\\
    doch alles in der Still’
    und wie es sich schicket.\\
    Mein Wunsch und Begehren
    kann niemand verwehren,
    es bleibet dabei:\\
    Die Gedanken sind frei!

    Und sperrt man mich ein
    im finsteren Kerker,\\
    das alles sind rein
    vergebliche Werke,\\
    denn meine Gedanken
    zerreißen die Schranken
    und Mauern entzwei.\\
    Die Gedanken sind frei!

    Drum will ich auf immer
    den Sorgen entsagen\\
    und will mich auch nimmer
    mit Grillen mehr plagen.\\
    Man kann ja im Herzen
    stets lachen und scherzen
    und denken dabei:\\
    Die Gedanken sind frei!

    Ich liebe den Wein,
    mein Mädchen vor allen,\\
    sie tut mir allein
    am besten gefallen.

    Ich bin nicht alleine
    bei meinem Glas Weine,
    mein Mädchen dabei.\\
    Die Gedanken sind frei!
    
\end{guitarMagic}
