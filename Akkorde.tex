\def\Titel{Akkorde}
\def\Interpret{}
\def\Referenz{}

\AkkordeSetup{}

\begin{figure}

    \subfloat{
        \begin{chord}
            \chordname{A}
            \single 4 2 2
            \single 3 2 3
            \single 2 2 4
            \nostrike 6
        \end{chord}
    \chordreset}
    %
    \subfloat{
        \begin{chord}
            \chordname{a}
            \single 2 1 2
            \single 4 2 3
            \single 3 2 4
            \nostrike 6
        \end{chord}
    \chordreset}

\end{figure}\begin{figure}
    
    \subfloat{
        \begin{chord}
            \chordname{H7}
            \single 4 1 2
            \single 5 2 3
            \single 3 2 4
            \single 1 2 5
            \nostrike 6
        \end{chord}
    \chordreset}
    %
    \subfloat{
        \begin{chord}
            \chordname{C}
            \single 2 1 2
            \single 4 2 3
            \single 5 3 4
            \nostrike 6
        \end{chord}
    \chordreset}
    
\end{figure}\begin{figure}
    
    \subfloat{
        \begin{chord}
            \chordname{D}
            \single 3 2 2
            \single 1 2 3
            \single 2 3 4
            \nostrike 5
            \nostrike 6
        \end{chord}
    \chordreset}
    %
    \subfloat{
        \begin{chord}
            \chordname{d}
            \single 1 1 2
            \single 3 2 3
            \single 2 3 4
            \nostrike 5
            \nostrike 6
        \end{chord}
    \chordreset}

\end{figure}\begin{figure}
    
    \subfloat{
        \begin{chord}
            \chordname{E}
            \single 3 1 2
            \single 5 2 3
            \single 4 2 4
        \end{chord}
    \chordreset}
    %
    \subfloat{
        \begin{chord}
            \chordname{e}
            \single 5 2 3
            \single 4 2 4
        \end{chord}
    \chordreset}
    
\end{figure}\begin{figure}
    
    \subfloat{
        \begin{chord}
            \chordname{G}
            \single 5 2 3
            \single 6 3 4
            \single 1 3 5
        \end{chord}
    \chordreset}
    
\end{figure}

%    Akkord-Notation:
%    Je Akkord ein \subfloat erstellen. \bar, \single und \nostrike können beliebig oft verwendet oder
%    auch weggelassen werden. Anstelle von Zahlen für die Finger können auch einzelne Buchstaben verwendet
%    werden. Das % zwischen den \subfloat-Abschnitten nicht entfernen!
%    Für Zeilenumbruch ein \end{figure}\begin{figure} einfügen.
%
%    \subfloat{
%        \def\chordfretstart{<erster anzuzeigender Bund>} % Optional, Standard: 1
%        \def\chordfretend{<letzter anzuzeigender Bund>}  % Optional, Standard: 4
%        \begin{chord}
%            \chordname{<Akkordname>}                     % Akkordname, z.B. \chordname{H7}
%            \bar <Start> <Ende> <Finger>                 % Barré beginnend bei hoher Saite, z.B. \bar 5 1 2
%            \single <Saitennummer> <Bund> <Finger>       % Einzelne Saite, z.B. \single 4 2 2
%            \nostrike <Saitennummer>                     % Nicht anzuschlagende Saite, z.B. \nostrike 6
%        \end{chord}
%    \chordreset}
%    %
%    \subfloat{
%        \begin{chord}
%            \chordname{A}
%            \single 4 2 2
%            \single 3 2 3
%            \single 2 2 4
%            \nostrike 6
%        \end{chord}
%    \chordreset}
%    %
%    \subfloat{
%       ...